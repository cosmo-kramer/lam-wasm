\documentclass[a4paper,11pt]{article}

\usepackage{fullpage}
\usepackage{mathpartir}
\usepackage{amsmath,amssymb}

\newcommand{\syntax}[1]{\ensuremath{\mathsf{#1}}}
\newcommand{\eref}[1]{\syntax{ref}\;#1}
\newcommand{\eassign}[2]{#1 \mathrel{:=} #2}
\newcommand{\ederef}[1]{\mathop{!} #1}
\newcommand{\eapp}[2]{#1 \; #2}
\newcommand{\elet}[3]{\syntax{let}\;#1 = #2 \mathrel{\syntax{in}} #3}


\newcommand{\syntax}[1]{\ensuremath{\mathsf{#1}}}
\newcommand{\eassert}[1]{\syntax{assert}(#1)}

\newcommand{\type}[1]{\ensuremath{\mathtt{#1}}}
\newcommand{\tun}{\type{Un}}
\newcommand{\tbase}{\type{b}}
\newcommand{\tbref}{\type{R}}
\newcommand{\tbool}{\type{bool}}
\newcommand{\tint}{\type{int}}
\newcommand{\tunit}{\type{unit}}
\newcommand{\tref}[2]{\type{ref}_{#2}\;#1}
\newcommand{\peq}[2]{#1 = #2}
\newcommand{\pun}[1]{\type{Un}(#1)}

\newcommand{\subtype}{\sqsubseteq}
\newcommand{\erase}[1]{|#1|}
\newcommand{\sep}{\;|\;}

\renewcommand{\check}{\Leftarrow}
\renewcommand{\infer}{\Rightarrow}

\title{Bidirectional type system with {\tun} and Refinement types}
\date{}

\begin{document}
\maketitle

\[
\begin{array}{llll}
%
\mbox{Refinements} & \phi & ::= & \peq{e_1}{e_2} ~|~ \pun{e} ~|~ \ldots  \\
  %
  \mbox{Base types} & \tbase & ::= & \tbool \sep \tint \sep \tunit \\
  %
\mbox{Ref.\ base types} & \tbref & ::= & \{x : \tbase ~|~ \phi\}\\
  
  \mbox{Types} & \tau & ::= & \tbref \sep \tau_1 \to \tau_2 \sep \tref{\tau}{x.\phi} \sep \tun  
\end{array}
\]

\[
\begin{array}{ll}
  \Gamma \vdash e \infer \tau	&\text{inference} \\
  \Gamma \vdash e \check \tau	&\text{checking}
\end{array}
\]

\begin{mathpar}
\hbox to \textwidth{\hfil Generally applicable rules\hfil}

  \inferrule{\Gamma(x) = \tau}{\Gamma \vdash x \infer \tau}

  \inferrule{
  \Gamma \vdash e \infer \tau_1 \\ \tau_1 \subtype \tau_2
  }{\Gamma \vdash e \check \tau_2}

  \inferrule{\Gamma \vdash e \check \tau}{\Gamma \vdash e : \tau \infer \tau}
\\\\
  \inferrule{
    \Gamma \vdash e_1 \infer \tau_1 \\
    \Gamma, x : \tau_1 \vdash e_2 \infer \tau_2
  }{\Gamma \vdash \elet{x}{e_1}{e_2} \infer \tau_2}

  \inferrule{
    \Gamma \vdash e_1 \infer \tau_1 \\
    \Gamma, x : \tau_1 \vdash e_2 \check \tau_2
  }{\Gamma \vdash \elet{x}{e_1}{e_2} \check \tau_2}
\end{mathpar}

\begin{mathpar}
  \begin{array}{c@{~~}|@{~~}c}
    \hline \\
    \mbox{Usual rule} & \mbox{{\tun} rule} \\
    \\
    \hline
    \\

  \inferrule{ }{\Gamma \vdash n \check \{x:\tint ~|~ \peq{x}{n}\}}
  &
  \inferrule{ }{\Gamma \vdash n \check \tun}
  \\ \\

  \inferrule{\Gamma, x: \tau_1 \vdash e \check \tau_2}{\Gamma \vdash \lambda x.e \check \tau_1 \to \tau_2}

  &
  
  \inferrule{\Gamma, x: \tun \vdash e \check \tun}{\Gamma \vdash \lambda x.e \check \tun}

  \\ \\

  \inferrule
      {
        \Gamma \vdash e_1 \infer \tau_1 \to \tau_2 \\
        \Gamma \vdash e_2 \check \tau_1
      }
      {
        \Gamma \vdash \eapp{e_1}{e_2} \infer \tau_2
      }
      
      &

   \inferrule
       {
         \Gamma \vdash e_1 \infer \tun \\
         \Gamma \vdash e_2 \check \tun
       }
       {\Gamma \vdash \eapp{e_1}{e_2} \infer \tun}

    \\ \\

    \inferrule
        {\Gamma \vdash e \check \{x:\tau ~|~ \phi\}}
        {\Gamma \vdash \eref{e} \check {\type{ref}_{x.\phi}\;{\tau}}}

        &

    \inferrule
        {\Gamma \vdash e \check \tun}
        {\Gamma \vdash \eref{e} \check \tun}

     \\ \\

     \inferrule
         {\Gamma \vdash e_1 \infer {\type{ref}_{x.\phi}\;{\tau}} \\ \Gamma \vdash e_2 \check \{x: \tau ~|~ \phi\}}
         {\Gamma \vdash \eassign{e_1}{e_2} \infer \tunit}

         &

     \inferrule
         {\Gamma \vdash e_1 \infer \tun \\ \Gamma \vdash e_2 \check \tun}
         {\Gamma \vdash \eassign{e_1}{e_2} \infer \tun}

      \\ \\
      
          \inferrule
        {\Gamma \vdash e \infer {\type{ref}_{x.\phi}\;{\tau}}}
        {\Gamma \vdash \ederef{e} \infer  \{x:\tau ~|~ \phi\} }

        &

    \inferrule
        {\Gamma \vdash e \infer \tun}
        {\Gamma \vdash \ederef{e} \infer \tun}

     \\ \\
     
        \inferrule
        {\Gamma \vdash \forall i, e_i \check {\type{\tau}}}
        {\Gamma \vdash Cons \quad e_i ... \infer  \{x: \tau_{Cons} ~|True\} }


     \\ \\

      \inferrule
          {\erase{\Gamma} \infer \phi}
          {\Gamma \infer \eassert{\phi}: \{\_ : \tunit ~|~ \phi\}}
          \\ \\

          \hline
  \end{array}
\end{mathpar}

The judgment $\Delta \vdash \phi$ is entailment in the refinement
logic. The erasure function $\erase{\Gamma}$ is defined as follows.
\[\begin{array}{lll}
\erase{\bullet} & = & \bullet \\
%
\erase{\Gamma, x: \{y: \tbase ~|~ \phi\}} & = & \erase{\Gamma}, \phi[x/y] \\
%
\erase{\Gamma, x: \tau} & = & \erase{\Gamma} ~~~\mbox{when }\tau \not= \tbase
\end{array}\]

\framebox{Subtyping}
\begin{mathpar}
\inferrule{ }{\tun \subtype \{x: \tbase ~|~ \pun{x}\}} \and
%
\inferrule{ }{\{x: \tbase ~|~ \pun{x}\} \subtype \tun} \and
%
\inferrule{\erase{\Gamma} \vdash \forall x. \phi \Rightarrow \phi'}
          {\Gamma \vdash \{x:\tau ~|~ \phi\} \subtype \{x:\tau ~|~ \phi'\}} \and
          %
\mbox{(Standard rules for subtyping are included)} 
\end{mathpar}
\end{document}
